%\documentstyle[epsf,twocolumn]{jarticle}       %LaTeX2.09仕様
\documentclass[twocolumn]{jarticle}     %pLaTeX2e仕様

%\usepackage[backend=bibtex, style=numeric]{biblatex}
%\addbibresource{sankou.bib}
%%%%%%%%%%%%%%%%%%%%%%%%%%%%%%%%%%%%%%%%%%%%%%%%%%%%%%%%%%%%%%
%%
%%  基本 バージョン
%%
%%%%%%%%%%%%%%%%%%%%%%%%%%%%%%%%%%%%%%%%%%%%%%%%%%%%%%%%%%%%%%%%
\setlength{\topmargin}{-45pt}
%\setlength{\oddsidemargin}{0cm}
\setlength{\oddsidemargin}{-7.5mm}
%\setlength{\evensidemargin}{0cm}
\setlength{\textheight}{24.1cm}
%setlength{\textheight}{25cm}
\setlength{\textwidth}{17.4cm}
%\setlength{\textwidth}{172mm}
\setlength{\columnsep}{11mm}

\setlength{\intextsep}{8pt}
\setlength{\textfloatsep}{8pt}
\setlength{\floatsep}{1pt}

\kanjiskip=.07zw plus.5pt minus.5pt



%【節がかわるごとに(1.1)(1.2) …(2.1)(2.2)と数式番号をつけるとき】
%\makeatletter
%\renewcommand{\theequation}{%
%\thesection.\arabic{equation}} %\@addtoreset{equation}{section}
%\makeatother

%\renewcommand{\arraystretch}{0.95} 行間の設定

\usepackage[dvipdfmx]{graphicx}   %pLaTeX2e仕様(\documentstyle ->\documentclass)
\usepackage[dvipdfmx]{color}
\usepackage{scalefnt}
\usepackage{bm}
\usepackage{here}
\usepackage{url}
\usepackage{amsmath}
\usepackage{amsfonts}
\usepackage[subrefformat=parens]{subcaption}
\captionsetup{compatibility=false}
%%%%%%%%%%%%%%%%%%%%%%%%%%%%%%%%%%%%%%%%%%%%%%%%%%%%%%%%
\usepackage{comment}
\usepackage{subcaption}
\usepackage{multirow}
\usepackage{nidanfloat}
%\usepackage[table,xcdraw]{xcolor}
\usepackage[dvipdfmx]{hyperref}

\usepackage[normalem]{ulem}
\useunder{\uline}{\ul}{}

\begin{document}

\twocolumn[
\noindent
\hspace{1em}

令和2年7月14日(水) ゼミ資料
\hfill
\ \ B4 高山 裕成

\vspace{2mm}
\hrule
\begin{center}
{\Large  進捗報告}
\end{center}
\hrule
\vspace{3mm}
]

% \footnotesize
\section{進捗}

\begin{itemize}
  \item \sout{コマ画像データの切り出し}
  \item \sout{コマ画像データのみでの感情推定}
  \item \sout{セリフの遷移情報を加味する}
  \item 発話者情報を加味する
  \item i2v の fine-tuning
\end{itemize}

\section{実験設定の変更}
全体的な実験設定変更として, エポック数を 200 から 50 に変更した. 精度への影響を確かめるため, セリフ 1 文の感情推定をやり直した. 識別器としては 3 層 MLP を用いた. 入力次元 768, 隠れ層次元 30, 出力次元 2 とした.
その結果を表\ref{table:result_1} に示す.

\section{コマ画像データのみでの感情推定}
識別器としては 3 層 MLP を用いた. 入力次元 4096, 隠れ層次元 768, 出力次元 2 とした.
その結果を表\ref{table:result_4} に示す.

\section{セリフ 1 文のマルチモーダル感情推定}
識別器としては 3 層 MLP を用いた. 入力次元 $768+4096$, 隠れ層次元 300, 出力次元 2 とした. BERT は最終層のみをチューニングした.
その結果を表\ref{table:result_3} に示す.

\section{裏でやっていること}
\begin{itemize}
  \item 過去のセリフを考慮した感情推定 やり直し (セリフの遷移情報を加味)
  \item コマベクトルを PCA で圧縮したものを用いたマルチモーダル感情推定
\end{itemize}

\section{タスク}
情報処理学会関西支部 資料作り

\newpage

\begin{table*}[!b]
\begin{center}
\caption{セリフ 1 文の感情推定}
\label{table:result_1}
\scalebox{0.61}{
\begin{tabular}{lccccccccccccccc|ccc}
\hline
\multicolumn{1}{c}{\multirow{2}{*}{}} & \multicolumn{3}{c}{ギャグ} & \multicolumn{3}{c}{少女漫画} & \multicolumn{3}{c}{少年漫画} & \multicolumn{3}{c}{青年漫画} & \multicolumn{3}{c|}{萌え} & \multicolumn{3}{c}{5 タッチ総合} \\
\multicolumn{1}{c}{} & Acc & P-Recall & P-F1 & Acc & P-Recall & P-F1 & Acc & P-Recall & P-F1 & Acc & P-Recall & P-F1 & Acc & P-Recall & P-F1 & Acc & P-Recall & P-F1 \\ \hline
BERT fixed & 0.712 & 0.200 & 0.174 & 0.567 & 0.632 & 0.623 & 0.766 & 0.083 & 0.118 & 0.692 & 0.643 & 0.474 & 0.594 & 0.591 & 0.500 & 0.666 & 0.510 & 0.473 \\
BERT fine tuning & {\ul 0.818} & 0.200 & {\ul 0.250} & {\ul 0.612} & 0.711 & {\ul 0.675} & 0.766 & 0.083 & 0.118 & {\ul 0.862} & 0.500 & {\ul 0.609} & {\ul 0.609} & 0.591 & {\ul 0.510} & {\ul 0.733} & 0.521 & {\ul 0.535} \\ \hline
ベースライン & \multicolumn{1}{l}{0.848} & 0 & 0 & 0.432 & 0 & 0 & 0.812 & 0 & 0 & 0.784 & 0 & 0 & 0.656 & 0 & 0 & 0.705 & 0 & 0
\end{tabular}
}
\end{center}
\end{table*}

\begin{table*}[!b]
\begin{center}
\caption{コマ画像データのみでの感情推定}
\label{table:result_4}
\scalebox{0.61}{
\begin{tabular}{lccccccccccccccc|ccc}
\hline
\multicolumn{1}{c}{\multirow{2}{*}{}} & \multicolumn{3}{c}{ギャグ} & \multicolumn{3}{c}{少女漫画} & \multicolumn{3}{c}{少年漫画} & \multicolumn{3}{c}{青年漫画} & \multicolumn{3}{c|}{萌え} & \multicolumn{3}{c}{5 タッチ総合} \\
\multicolumn{1}{c}{} & Acc & P-Recall & P-F1 & Acc & P-Recall & P-F1 & Acc & P-Recall & P-F1 & Acc & P-Recall & P-F1 & Acc & P-Recall & P-F1 & Acc & P-Recall & P-F1 \\ \hline
koma only & 0.697 & 0.100 & 0.091 & 0.612 & 0.526 & 0.606 & 0.438 & 0.750 & 0.333 & 0.477 & 0.571 & 0.320 & 0.594 & 0.364 & 0.381 & 0.564 & 0.479 & 0.393 \\ \hline
ベースライン & \multicolumn{1}{l}{0.848} & 0 & 0 & 0.432 & 0 & 0 & 0.812 & 0 & 0 & 0.784 & 0 & 0 & 0.656 & 0 & 0 & 0.705 & 0 & 0
\end{tabular}
}
\end{center}
\end{table*}


\begin{table*}[!b]
\begin{center}
\caption{コマ画像データのみでのマルチモーダル感情推定}
\label{table:result_3}
\scalebox{0.61}{
\begin{tabular}{lccccccccccccccc|ccc}
\hline
\multicolumn{1}{c}{\multirow{2}{*}{}} & \multicolumn{3}{c}{ギャグ} & \multicolumn{3}{c}{少女漫画} & \multicolumn{3}{c}{少年漫画} & \multicolumn{3}{c}{青年漫画} & \multicolumn{3}{c|}{萌え} & \multicolumn{3}{c}{5 タッチ総合} \\
\multicolumn{1}{c}{} & Acc & P-Recall & P-F1 & Acc & P-Recall & P-F1 & Acc & P-Recall & P-F1 & Acc & P-Recall & P-F1 & Acc & P-Recall & P-F1 & Acc & P-Recall & P-F1 \\ \hline
multi & 0.773 & 0.200 & 0.211 & 0.687 & 0.763 & 0.734 & 0.703 & 0.417 & 0.345 & 0.769 & 0.643 & 0.545 & 0.641 & 0.500 & 0.489 & 0.715 & 0.583 & 0.546 \\ \hline
ベースライン & \multicolumn{1}{l}{0.848} & 0 & 0 & 0.432 & 0 & 0 & 0.812 & 0 & 0 & 0.784 & 0 & 0 & 0.656 & 0 & 0 & 0.705 & 0 & 0
\end{tabular}
}
\end{center}
\end{table*}


\end{document}
