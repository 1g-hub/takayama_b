%\documentstyle[epsf,twocolumn]{jarticle}       %LaTeX2.09仕様
\documentclass[twocolumn]{jarticle}     %pLaTeX2e仕様

%\usepackage[backend=bibtex, style=numeric]{biblatex}
%\addbibresource{sankou.bib}
%%%%%%%%%%%%%%%%%%%%%%%%%%%%%%%%%%%%%%%%%%%%%%%%%%%%%%%%%%%%%%
%%
%%  基本 バージョン
%%
%%%%%%%%%%%%%%%%%%%%%%%%%%%%%%%%%%%%%%%%%%%%%%%%%%%%%%%%%%%%%%%%
\setlength{\topmargin}{-45pt}
%\setlength{\oddsidemargin}{0cm}
\setlength{\oddsidemargin}{-7.5mm}
%\setlength{\evensidemargin}{0cm}
\setlength{\textheight}{24.1cm}
%setlength{\textheight}{25cm}
\setlength{\textwidth}{17.4cm}
%\setlength{\textwidth}{172mm}
\setlength{\columnsep}{11mm}

\setlength{\intextsep}{8pt}
\setlength{\textfloatsep}{8pt}
\setlength{\floatsep}{1pt}

\kanjiskip=.07zw plus.5pt minus.5pt



%【節がかわるごとに(1.1)(1.2) …(2.1)(2.2)と数式番号をつけるとき】
%\makeatletter
%\renewcommand{\theequation}{%
%\thesection.\arabic{equation}} %\@addtoreset{equation}{section}
%\makeatother

%\renewcommand{\arraystretch}{0.95} 行間の設定

\usepackage[dvipdfmx]{graphicx}   %pLaTeX2e仕様(\documentstyle ->\documentclass)
\usepackage{scalefnt}
\usepackage{bm}
\usepackage{here}
\usepackage{url}
\usepackage{amsmath}
\usepackage{amsfonts}
\usepackage[subrefformat=parens]{subcaption}
\captionsetup{compatibility=false}
%%%%%%%%%%%%%%%%%%%%%%%%%%%%%%%%%%%%%%%%%%%%%%%%%%%%%%%%
\usepackage{comment}
\usepackage{subcaption}
\usepackage{multirow}
\usepackage{nidanfloat}
\usepackage[dvipdfmx]{hyperref}

\usepackage[normalem]{ulem}
\useunder{\uline}{\ul}{}

\begin{document}

\twocolumn[
\noindent
\hspace{1em}

令和2年6月10日(水) ゼミ資料
\hfill
\ \ B4 高山 裕成

\vspace{2mm}
\hrule
\begin{center}
{\Large  進捗報告}
\end{center}
\hrule
\vspace{3mm}
]

\section{あらすじ}
研究から逃げた.

% \footnotesize
\section{進捗}

\begin{itemize}
  \item 未知語率再測定
\end{itemize}

\section{データセットに含まれる未知語率}

BERT の事前学習済モデル\footnote{http://nlp.ist.i.kyoto-u.ac.jp}のボキャブラリーに含まれているかどうかを各タッチについて拡張前後で算出した. また, transformers ライブラリ内のトークナイザでサブワードに分割した場合も算出した. 用いている BERT のモデルは形態素をサブワードに分割したものを基本単位として事前学習しているので, 後者の方が有用である可能性が高い.

\subsubsection{形態素(Juman++)}
従来, 実験で用いていた手法における未知語率を表したのが表\ref{tab:unknown} である.
拡張前で約 25\%, 拡張後で約 68\% であった.

\begin{table}[htb]
\begin{center}
\caption{データセットに含まれる未知語率}
\scalebox{0.8}{
\begin{tabular}{llccccc}
\hline
\multicolumn{2}{c}{\multirow{2}{*}{}} & \multirow{2}{*}{ギャグ} & \multirow{2}{*}{少女} & \multirow{2}{*}{少年} & \multirow{2}{*}{青年} & \multirow{2}{*}{萌え系} \\
\multicolumn{2}{c}{} &  &  &  &  &  \\ \hline
拡張前 & 総単語数 & 270 & 289 & 274 & 276 & 276 \\
 & 未知語率 & 0.244 & 0.263 & 0.230 & 0.236 & 0.232 \\ \hline
拡張後 & 総単語数 & 3030 & 3209 & 3089 & 3154 & 3154 \\
 & 未知語率 & 0.681 & 0.688 & 0.681 & 0.682 & 0.682
\end{tabular}
\label{tab:unknown}
}
\end{center}
\end{table}

\subsubsection{サブワード化}


\begin{table}[htb]
\begin{center}
\caption{データセットに含まれる未知語率(サブワード化)}
\scalebox{0.8}{
\begin{tabular}{llccccc}
\hline
\multicolumn{2}{c}{\multirow{2}{*}{}} & \multirow{2}{*}{ギャグ} & \multirow{2}{*}{少女} & \multirow{2}{*}{少年} & \multirow{2}{*}{青年} & \multirow{2}{*}{萌え系} \\
\multicolumn{2}{c}{} &  &  &  &  &  \\ \hline
拡張前 & 総単語数 & 311 & 331 & 315 & 316 & 316 \\
 & 未知語率 & 0.026 & 0.021 & 0.019 & 0.022 & 0.025 \\ \hline
拡張後 & 総単語数 & 2633 & 2728 & 2664 & 2705 & 2708 \\
 & 未知語率 & 0.090 & 0.089 & 0.088 & 0.089 & 0.089
\end{tabular}
\label{tab:unknown_bert_tokenize}
}
\end{center}
\end{table}

\subsection{未知語 (ギャグオリジナル サブワード)}
\begin{verbatim}
'A', '?', 'DVD',
'B', '!', 'GPU',
',', '.'
\end{verbatim}

対策として, 句読点はストップワードとして除外していたが, `.' や `,' も除外しておく.
`A' や `B' はキャラクタ由来のワードであるから, 辞書内に含まれる固有名詞で置換する.
他のタッチのオリジナルデータでは `2' など数字も未知語として表れていた. この対策としては 数値を表すトークンまたは `0' に置換する.

\subsection{未知語の品詞別頻度}
萌えタッチについて, 拡張前後・サブワードにするかしないかの 4 パターンについて,
未知語を MeCab を使って品詞解析し, 品詞ごとに頻度をまとめたものが表\ref{tab:freq} である.

\begin{table}[htb]
\begin{center}
\caption{未知語の品詞別頻度(萌えタッチ)}
\scalebox{0.7}{
\begin{tabular}{lcc|cc}
moe & オリジナル & 拡張後 & オリジナル(subword) & 拡張後(subword) \\ \hline
名詞 & 42 & {\ul 1927} & 8 & {\ul 258} \\
形容詞 & 2 & 186 & 0 & 24 \\
副詞 & 2 & 202 & 0 & 15 \\
動詞 & 14 & {\ul 495} & 0 & 27 \\
助動詞 & 14 & 93 & 0 & 10 \\
助詞 & 9 & 195 & 0 & 33 \\
連体詞 & 2 & 15 & 0 & 0 \\
接頭詞 & 0 & 89 & 0 & 7 \\
フィラー & 1 & 6 & 0 & 0 \\
接続詞 & 1 & 4 & 0 & 1 \\
感動詞 & 5 & 7 & 0 & 0 \\
記号 & 0 & 9 & 0 & 1
\end{tabular}
\label{tab:freq}
}
\end{center}
\end{table}

\newpage

\section{今後の実験予定}
\begin{itemize}
  \item サブワード化した状態で前回と同じ実験を回す.
  \item 直前 n - 1 文 を考慮した n 文を入力して末尾入力の感情推定をする.
\end{itemize}

\bibliographystyle{unsrt}
\bibliography{sankou}


\end{document}
