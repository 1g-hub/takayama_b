\documentclass[a4paper,twoside,twocolumn,10pt]{article}
\usepackage{abstract} % Style for abstracts in Dept. CSIS, OPU
%\usepackage{abstract_past} % Style for abstracts for the past curriculum

%%%%%%%%%% Designate packages you need %%%%%%%%%%
\usepackage{graphicx} % Enhanced support for graphics
\usepackage{url} % Verbatim with URL-sensitive line breaks

%%%%%%%%%% Parameters that should be customized %%%%%%%%%%
% Language (1 = Japanese, 2 = English)
\setlang{2}
% Bachelor or Master (1 = Bachelor, 2 = Master)
\setborm{2}
% Fiscal year
\setfy{2015}
% Group number
\setgnum{123}
% Presentation order
\setorder{2}
% Increase page number (optional)
%% \pplus{1}

% Title
\title{Template for Bachelor's Thesis Abstract and Master's Thesis Abstract}
% Author
\author{Taro Joho}
%%%%%%%%%% Parameters that should be customized (end) %%%%%%%%%%

\begin{document}
\maketitle % Insert title
\small

\section{Introduction}
This article explains usage of the
template for Bachelor's Thesis Abstract at Computer Science Course,
College of Engineering, Osaka Prefecture University
(also at Department of Computer Science and Intelligent Systems,
School of Engineering, Osaka Prefecture University)
and
Master's Thesis Abstract at Department of Computer Science and
Intelligent Systems, Osaka Prefecture University.
This article itself is made using the template.


\section{Guidelines}

\subsection{Bachelor's Thesis Abstract}
\begin{enumerate}
\item Paper Size and Page Limit\\
  In principle, Bachelor's thesis abstract should be one page long.
  If there is a special reason, it can be extended up to two pages.
  The abstracts should be printed in both sides of A4 papers
  in the presentation order in the group.
\item Format\\
  The abstract should be written using
  designated LaTeX or MS Word template.
\item Group Number and Page Number\\
  In the upper right of the page(s),
  ``(Group number) - (Page number in the group)''
  should be printed.
\end{enumerate}

\subsection{Master's Thesis Abstract}
\begin{enumerate}
\item Paper Size and Page Limit\\
  Master's thesis abstract should be two pages long.
  It should be printed on both sides of an A4 paper.
\item Format\\
  The abstract should be written using
  designated LaTeX or MS Word template.
\item Group Number and Page Number\\
  In the upper right of the pages,
  ``(Group number) - (Page number in the group)''
  should be printed.
\end{enumerate}

\section{LaTeX Template}
\subsection{Files}
\begin{itemize}
\item abstract\_en.pdf\\
  Sample PDF (English version)
\item abstract\_en.tex\\
  Sample LaTeX file (English version)
\item index\_en.bib\\
  Sample BibTeX file (English version)
\item abstract.sty\\
  Style for abstracts in Dept. CSIS
\item abbrvunsrt.bst\\
  BibTeX style file (English version)
\item CSIS.eps\\
  Logo of department as a figure example
\item fancyhdr.sty\\
  Style file for extensive control of page headers and footers
  downloaded from \url{https://www.ctan.org/pkg/fancyhdr}
\item titlesec.sty\\
  Style file to select alternative section titles
  downloaded from \url{https://www.ctan.org/pkg/titlesec}
\end{itemize}

\subsection{Setup}
In the beginning of \textit{abstract.tex},
the following parameters are available.
They should be set by yourself.
%
\begin{verbatim}
% Language (1 = Japanese, 2 = English)
\setlang{2}
% Bachelor or Master (1 = Bachelor, 2 = Master)
\setborm{2}
% Fiscal year
\setfy{2015}
% Group number
\setgnum{3}
% Presentation order
\setorder{2}
% Increase page number (optional)
%% \pplus{1}
\end{verbatim}
%
The \textit{Presentation order} represents the order of presentation in the group.
According to it, the page numbers are automatically calculated based on
the assumption that 1 page for each person in the Bachelor's thesis
abstract and 2 pages in the Master's thesis abstract.
If necessary, the page number can be modified by adding (or reducing)
a certain number, which is designated by the following command
$\backslash$pplus\{1\} (for adding 1).

\subsection{Figures and Tables}
Table~\ref{tbl:kuku} and Figure~\ref{fig:CSIS_logo}
are examples of a table and a figure.

\begin{table}[tb]
  \caption{Table example: Multiplication table.}
  \label{tbl:kuku}
  \centering
  \begin{tabular}{|c||c|c|c|c|c|c|c|c|c|} \hline
    - &  1 &  2 &  3 &  4 &  5 &  6 &  7 &  8 &  9 \\ \hline \hline
    1 &  1 &  2 &  3 &  4 &  5 &  6 &  7 &  8 &  9 \\ \hline
    2 &  2 &  4 &  6 &  8 & 10 & 12 & 14 & 16 & 18 \\ \hline
    3 &  3 &  6 &  9 & 12 & 15 & 18 & 21 & 24 & 27 \\ \hline
    4 &  4 &  8 & 12 & 16 & 20 & 24 & 28 & 32 & 36 \\ \hline
    5 &  5 & 10 & 15 & 20 & 25 & 30 & 35 & 40 & 45 \\ \hline
    6 &  6 & 12 & 18 & 24 & 30 & 36 & 42 & 48 & 54 \\ \hline
    7 &  7 & 14 & 21 & 28 & 35 & 42 & 49 & 56 & 63 \\ \hline
    8 &  8 & 16 & 24 & 32 & 40 & 48 & 56 & 64 & 72 \\ \hline
    9 &  9 & 18 & 27 & 36 & 45 & 54 & 63 & 72 & 81 \\ \hline
  \end{tabular}
\end{table}

\begin{figure}[tb]
  \centering
  \includegraphics[width=.3\hsize]{CSIS.eps}
  \caption{Figure example: Logo.}
  \label{fig:CSIS_logo}
\end{figure}

\subsection{References}
BibTeX is recommended to use with \textit{abbrvunsrt.bst}
bundled in the package.
It is obtained by disabling sorting in \textit{abbrv.bst}.
\cite{SakaiMe, Food, Neko} are examples.

\subsection{For Students in Old Curriculum}
Students who belong to Department of Computer Science and Intelligent Systems,
School of Engineering should use
\textit{abstract.sty} instead of \textit{abstract4past.sty}.

\section{MS Word Template}
MS Word version of the template that provides a similar format to the LaTeX version is available.

\subsection{Files}
\begin{itemize}
\item abstract\_en\_word.pdf\\
  Sample PDF (English version)
\item abstract\_en\_word.docx\\
  Sample Word file (English version)
\end{itemize}

\subsection{Header Setup}
Double-click inside the header area (near the top of the page) to edit the header.
Then, edit the following items.

\subsubsection{Paper Type}
\begin{itemize}
\item For Computer Science Course, College of Engineering\\
Use ``Bachelor's Thesis Abstract at Computer Science Course.''
\item For Department of Computer Science and Intelligent Systems\\
Use ``Master's Thesis Abstract at Department of Computer Science and Intelligent Systems.''
\item For Department of Computer Science and Intelligent Systems,
School of Engineering (old curriculum)\\
Use ``Bachelor's Thesis Abstract at Department of Computer Science and Intelligent Systems.''
\end{itemize}
\subsubsection{Fiscal Year and Group Number}
Directly edit fiscal year and your group number.

\subsubsection{Page Number}
Open ``Page Number Format'' and edit the number of ``Start at.''

\subsection{Styles}
Styles of Normal, Title, Author, Section, SubSection, SubSubSection, References, Table, Verbatim, Enumerate and Itemize are defined.
Use them when necessary.

%%%%%%%%%%%%%%%%%%%%%%%%%%%%%%

\bibliographystyle{abbrvunsrt}
\bibliography{index_en}
\end{document}
