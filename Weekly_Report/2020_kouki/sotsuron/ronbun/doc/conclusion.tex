\newpage
\changeindent{0cm}
\section{まとめと今後の課題}
\changeindent{2cm}

本研究では BERT で得たセリフの分散表現からセリフの感情を推定した. 実験結果から口語的なコーパスを用いて事前学習された hottoSNS-BERT の方が漫画のセリフのより合理的な分散表現が得られることを確認した. また, コマの画像情報も考慮したマルチモーダルな感情推定手法を提案し, 実験的にその有効性を確認した. しかし, データの扱い方やネットワークの構築について更なる工夫が必要であると分かった.

今後の課題としては, 以下の点が挙げられる.

\begin{itemize}
  \item コマベクトルの fine-tuning
  \item コマ画像の特徴的要素を付加したマルチモーダルな感情推定
  \item 最適なマルチモーダル特徴量の結合構造の探索
  \item 4 コマ漫画ストーリーデータセットの更なる拡張
  \item Manga109 やその他データセットを併用した半教師あり学習
\end{itemize}
