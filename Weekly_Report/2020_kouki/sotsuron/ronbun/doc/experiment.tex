\newpage
\changeindent{0cm}
\section{数値実験}
\changeindent{2cm}
本章では,実験について説明する.
\subsection{実験概要}
BERT および MLP を用いて新聞記事データの段落間の接続詞の有無を推定する実験をした.
\changeindent{0cm}
\subsection{データセット}
\subsubsection{使用データ}
本稿では叙述的な文章として毎日新聞データセット
\footnote{
http:\slash\slash{}www.nichigai.co.jp\slash{}sales\slash{}mainichi\slash{}mainichi-data.html
}
の新聞記事を用いた.
このデータセットにはジャンルごとに 2008 年から 2012 年までの記事がある.
そのなかのジャンルが国際のもので本文が 10 行以上ある 5000 記事を用いた.
それぞれの文章に対し, ``■'' や ``◇'' , また感嘆符といった記号を除去し,
``<>'' や ``《》'' の間に書かれる注釈や作者名等を除いた.
その上で, データセットの性質上, 数字の羅列などを含む記事や, 箇条書された記事が含まれているので,
そのような記事を取り除き, 文章として乱れていない記事を使用データとして扱った.

\subsection{実験準備}
実験では新聞記事データの 1 から 3 面の本文を用いた.
まず, 新聞記事データの各本文を段落ごとに分けて, すべて分かち書きする.
次に, はじめの段落を除いて, 各段落のはじめに接続詞があるものについてはその接続詞を ``[MASK]''
に変換する. そうでないものは段落の始めに ``[MASK]'' という単語を追加する.
続いて, 接続詞の直後に ``、'' があるものは, それを取り除く ( 接続詞があることが容易に推定できてしまうため ).
前後 2 段落を ``[SEP]'' でつなげたものを 1 データとして扱う. これらのトークンを含めて単語数が 248 個より多いデータを除く. \par
これにより各データには 2 段落あり, それらは ``[SEP]'' でつながっている. さらに, その直後には必ず ``[MASK]'' がある.
その ``[MASK]'' は, 通常の ``[MASK]'' のように, 接続詞を隠したものもあれば, ダミー ( その部分には単語は入らない ) もある.
\subsection{実験}
実験準備で得られた各データに対して, そのデータの 2 段落間の接続詞の有無を推定した.
今回は, データの不均等性に対してはデータ数を 1:1にすることで対応した.
また, 今回の実験では, 前回までで精度の良かった ``[MASK]'' 部分の分散表現に着目する方法のみをした.
まず, 先程のデータを BERT の入力として, 得られた各単語ベクトルから ``[MASK]'' 部分の分散表現を得る.
それを 3 層 MLP によって 2 次元にして, 接続詞の有無を推定した.
表 \ref{nn} に実験時のパラメータを示す.
学習は BERT の最終層および MLPに対してした.
学習率および最適化アルゴリズムは optuna によって調整した.
\begin{table}[ht] %MLP
	\begin{center}
		\caption{実験時のパラメータ}
		\label{nn}
		\begin{tabular}{|c|c|} \hline
			パラメータ			& 値						\\ \hline \hline
			入力層の次元数			& 768					\\ \hline
			隠れ層のノード数		& 768 					\\ \hline
			出力層の次元数			& 2 						\\ \hline
			バッチサイズ			& 2 						\\ \hline
			%{'optimizer': 'Adam', 'lr_bert': 0.00010574804290305164, 'lr_mlp': 4.541587759586735e-05}
			BERT の学習率					& 0.000105748		\\ \hline
			MLP の学習率					& 0.00004541588					\\ \hline
			最適化アルゴリズム     & Adam         \\ \hline
			活性化関数(隠れ層)		& ReLU 					\\ \hline
			活性化関数(出力層)		& Softmax function 		\\ \hline
			目的関数				& categorical cross entropy 	\\ \hline
			学習終了条件			& 2 epoch 		\\ \hline
		\end{tabular}
	\end{center}
\end{table}
\subsection{実験結果}
データ数があまり多いわけではないので, 5 分割検証を行い, その際の精度の平均値, 標準偏差を比較した.
また, ベースラインは, すべてをランダムに選択した際の期待値とした.
表 \ref{kekka} に 5 分割交差検証をしたときの平均及び標準偏差を示す.
\begin{table}[ht] %MLP
	\begin{center}
		\caption{交差検証の結果}
		\label{kekka}
		\begin{tabular}{|c|c|} \hline
			パラメータ			& 値						\\ \hline \hline
			正解率		& 0.7502 (0.0176)					\\ \hline
			F 値		& 0.6656 (0.06164)				\\ \hline
		\end{tabular}
	\end{center}
\end{table}
