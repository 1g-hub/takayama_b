\newpage
\changeindent{0cm}
\section{数値実験}
\changeindent{2cm}

本章では,数値実験について説明する.

\changeindent{0cm}
\subsection{実験概要}
\changeindent{2cm}

本研究では, 人工知能を用いた漫画の内容理解のために,
まず自然言語処理を用いて漫画のキャラクタのセリフの感情を推定する.
その上で漫画のコマの画像情報を加えたマルチモーダルな推定手法について検討する.
そして, 実験結果からセリフの感情推定とマルチモーダル化の精度への影響について考察した.

\changeindent{0cm}
\subsection{使用データ}
\changeindent{2cm}

4 コマ漫画ストーリーデータセットには 7 種類の感情ラベル(ニュートラル, 驚愕, 喜楽, 恐怖, 悲哀, 憤怒, 嫌悪)と, アノテーション不備によるラベル不明 (以下, ``UNK" とする) の全 8 種類
が含まれているが, 今回はこの ``UNK" のデータは除いた全 7 種類の感情ラベルが付いたデータのみを扱った. 表 \ref{table:data_ori} に各タッチのオリジナルデータに対する感情ラベルごとのデータ数を示す.

\changeindent{0cm}
\subsection{実験設定}
\changeindent{2cm}

データ数と解析の難しさの問題から, 本研究では 7 種類ある感情ラベルのうち,
喜楽のみを正例クラス, その他の感情ラベルをすべて負例クラスとした
2 クラスのセリフの感情推定を行った.

訓練用データは各タッチの前半 1 話から 5 話までの
Data Augmentation によって拡張されたセリフを用いて, 学習時には $20\%$ を検証用データとした.
また, 評価用モデルは検証用データにおける正例の F 値が最大となるエポックでのパラメータを採用し,
評価用データは後半 6 話から 10 話におけるオリジナルのセリフのみを用いた. そして, 各タッチに対してモデルを作成し評価を行った.
表 \ref{table:data_exp} に各実験で用いた正例と負例のデータ数を示す.



\begin{table}[!ht]
\begin{center}
\caption{オリジナルデータ数} %タイトルをつける
\label{table:data_ori} %ラベルをつけ図の参照を可能にする
\begin{tabular}{lccccc|l}
\hline
\multirow{2}{*}{感情ラベル} & \multirow{2}{*}{ギャグ} & \multirow{2}{*}{少女} & \multirow{2}{*}{少年} & \multirow{2}{*}{青年} & \multirow{2}{*}{萌え} & \multirow{2}{*}{合計} \\
                       &                      &                     &                     &                     &                     &                     \\ \hline
喜楽                     & 25                   & 77                  & 27                  & 33                  & 47                  & 209 (31.7\%)        \\ \hline
ニュートラル                 & 43                   & 8                   & 55                  & 33                  & 30                  & 169 (25.6\%)        \\
悲哀                     & 25                   & 12                  & 13                  & 16                  & 13                  & 79 (11.9\%)         \\
恐怖                     & 6                    & 11                  & 8                   & 8                   & 9                   & 42 (6.3\%)          \\
憤怒                     & 4                    & 5                   & 2                   & 7                   & 2                   & 20 (3.0\%)          \\
嫌悪                     & 2                    & 4                   & 3                   & 3                   & 4                   & 16 (2.4\%)          \\ \hline
UNK                    & 7                    & 3                   & 5                   & 2                   & 6                   & 23 (3.4\%)          \\ \hline
合計                     & 131                  & 136                 & 130                 & 131                 & 131                 & 659
\end{tabular}
\end{center}
\end{table}

\begin{table}[!ht]
\begin{center}
\caption{実験で用いたデータ数} %タイトルをつける
\label{table:data_exp} %ラベルをつけ図の参照を可能にする
\begin{tabular}{llccccc|l}
\hline
\multirow{2}{*}{}       & \multirow{2}{*}{感情ラベル} & \multirow{2}{*}{ギャグ} & \multirow{2}{*}{少女} & \multirow{2}{*}{少年} & \multirow{2}{*}{青年} & \multirow{2}{*}{萌え} & \multirow{2}{*}{合計} \\
                        &                        &                      &                     &                     &                     &                     &                     \\ \hline
\multirow{2}{*}{訓練用データ} & 喜楽                     & 1115                 & 2672                & 940                 & 999                 & 1766                & 7492 (37.1\%)       \\
                        & その他                    & 2766                 & 1396                & 3077                & 3146                & 2324                & 12709 (62.9\%)      \\ \hline
\multirow{2}{*}{評価用データ} & 喜楽                     & 10                   & 38                  & 12                  & 14                  & 22                  & 96 (29.5\%)         \\
                        & その他                    & 56                   & 29                  & 52                  & 51                  & 42                  & 230 (70.5\%)
\end{tabular}
\end{center}
\end{table}


\changeindent{0cm}
\subsection{実験 1 : セリフ 1 文のみを入力とする感情推定}
\changeindent{2cm}

実験 1 では, セリフ 1 文のみを入力とする感情推定を行った.
まず, セリフ 1 文を BERT に入力し 768 次元のセリフベクトルを得る.
それを識別器への入力として, ``喜楽" か ``その他" かを識別結果として推定した.
識別器としては 3 層 MLP を用いた.

本研究では, 各実験において Image Embedding 層には illustration2vec の筆者らによって公開されている事前学習済みモデルを用い, Text Embedding 層には 2 つの 事前学習済み BERT モデルを用いて精度を比較した.

\newpage

一つは, 京都大学が公開している日本語 Wikipedia より全 1800 万文を用いて事前学習させたモデル \cite{kyoto-bert} (以下, ``京大 BERT" と呼ぶ) , もう一つは hottolink 社が公開している大規模日本語 SNS コーパスを用いて事前学習させたモデル, hottoSNS-BERT \cite{hottoSNS-bert} を用いた. BERT モデルは最終層のみを fine-tuning した.

表 \ref{table:mlp_para} に 3 層 MLP で用いたパラメータ, そして表 \ref{table:ex_para} に学習で用いたパラメータを示す. 学習率は Optuna \cite{optuna_2019} によって最適なパラメータを探索した.
また, 多くのタッチにおいて, 正例は負例に対してデータ数が非常に少ない不均衡データを扱うことから, このまま学習してしまうと上手く学習できないという問題がある. 本研究ではこの解決策として, 損失関数のクラス重みとして, 訓練用データの各ラベルのデータ数の逆数を正規化したものを用いた. 表 \ref{table:ex_class_weight} に損失関数におけるタッチごとのクラス重みを示す.

\begin{table}[!h]
\vspace{20mm}
\caption{各実験における MLP のパラメータ}
\label{table:mlp_para}
\centering
\scalebox{1.35}{
\begin{tabular}{c|cc}
\hline
パラメータ & 実験 1 & 実験 2 \\ \hline
入力層次元 & 768 & 4864 \\
隠れ層次元 & \multicolumn{2}{c}{30} \\
出力層次元 & \multicolumn{2}{c}{2} \\ \hline
活性化関数 & \multicolumn{2}{c}{tanh} \\ \hline
ドロップアウト率 & \multicolumn{2}{c}{0.5}
\end{tabular}
}
\end{table}

\newpage

\begin{table}[!h]
\vspace{10mm}
\caption{学習パラメータ}
\label{table:ex_para}
\centering
\scalebox{1.35}{
\begin{tabular}{c|c|l}
\hline
パラメータ  & \multicolumn{2}{c}{実験 $1・2$}           \\ \hline
エポック数  & \multicolumn{2}{c}{50}                 \\ \hline
バッチサイズ & \multicolumn{2}{c}{16}                 \\ \hline
損失関数   & \multicolumn{2}{c}{Cross Entropy Loss\footnotetext[1]{出力層では softmax}} \\ \hline
最適化手法  & \multicolumn{2}{c}{Adam}
\end{tabular}
}
\vspace{10mm}
\end{table}

\begin{table}[!h]
\vspace{10mm}
\caption{損失関数におけるクラス重み}
\label{table:ex_class_weight}
\centering
\scalebox{1.35}{
\begin{tabular}{c|l|l}
\hline
タッチ                          & \multicolumn{1}{c|}{正例} & \multicolumn{1}{c}{負例} \\ \hline
ギャグタッチ                       & 0.773                   & 0.227                  \\
少女漫画タッチ                      & 0.272                   & 0.728                  \\
\multicolumn{1}{l|}{少年漫画タッチ} & 0.831                   & 0.169                  \\
青年漫画タッチ                      & 0.824                   & 0.176                  \\
萌えタッチ                        & 0.594                   & 0.406
\end{tabular}
}
\end{table}

\newpage
\changeindent{0cm}
\subsection{実験 1 : 結果}
\changeindent{2cm}

表 \ref{table:result_1} に評価用データに対する実験 1 の結果を示す. 表 \ref{table:result_1} における P-Recall, P-F 値はそれぞれ正例の再現率, F 値を表し, Acc は全体の精度を表している.
本研究では, ベースラインは前述したように, 正例が負例に対してデータ数が
非常に少ない不均衡データであることから, すべての予測値が負例の場合を設定した.
また, 図 \ref{fig:ex1_graph} に Text Embedding 層に hottoSNS-BERT を用いた場合のギャグタッチにおける学習曲線を示す. (左から Acc, 正例の F 値, loss を表し, 青線が訓練用データ, オレンジの線が検証用データを表す.)

表 \ref{table:result_1} より, Acc に関してはどちらの BERT モデルを用いても
ベースラインを超えることができた. また, すべての評価指標において京大 BERT よりも hottoSNS-BERT の方が上回った. 図 \ref{fig:ex1_graph} において, 訓練用データと検証用データの学習曲線の推移が酷似しているのは, データの切り分け方に問題があると推測される. 前述したように, 4 コマ漫画ストーリーデータセットには現状, 同一のストーリーを原則的に 4 コマ目がオチとなる ``一般" と逆に 1 コマ目にオチを持ってきた ``出オチ" の 2 つの構造のみからなっており, セリフや感情ラベルに大きな差はない. そのため, 検証用データをランダムにサンプリングしたことでデータの分布にも差が生まれなかったと考えられる.

\begin{table}[!hb]
\begin{center}
\caption{評価用データに対する実験 1 の結果}
\label{table:result_1}
\scalebox{0.45}{
\begin{tabular}{lccccccccccccccc|ccc}
\hline
\multicolumn{1}{c}{\multirow{2}{*}{}} & \multicolumn{3}{c}{ギャグ}                            & \multicolumn{3}{c}{少女漫画}             & \multicolumn{3}{c}{少年漫画}             & \multicolumn{3}{c}{青年漫画}             & \multicolumn{3}{c|}{萌え}               & \multicolumn{3}{c}{5 タッチ総合}         \\
\multicolumn{1}{c}{}                  & Acc                       & P-Recall & P-F 値       & Acc         & P-Recall & P-F 値       & Acc         & P-Recall & P-F 値       & Acc         & P-Recall & P-F 値       & Acc         & P-Recall & P-F 値       & Acc         & P-Recall & P-F 値       \\ \hline
京大 BERT                               & {\ul 0.818}               & 0.200    & 0.250       & 0.612       & 0.711    & 0.675       & 0.766       & 0.083    & 0.118       & {\ul 0.862} & 0.500    & 0.609       & 0.609       & 0.591    & 0.510       & 0.733       & 0.521    & 0.535       \\
hottoSNS-BERT                         & {\ul 0.818}               & 0.300    & {\ul 0.333} & {\ul 0.627} & 0.816    & {\ul 0.713} & {\ul 0.813} & 0.167    & {\ul 0.250} & 0.846       & 0.643    & {\ul 0.643} & {\ul 0.688} & 0.500    & {\ul 0.524} & {\ul 0.758} & 0.583    & {\ul 0.586} \\ \hline
ベースライン                                & \multicolumn{1}{l}{0.848} & -        & -           & 0.432       & -        & -           & 0.812       & -        & -           & 0.784       & -        & -           & 0.656       & -        & -           & 0.705       & -        & -
\end{tabular}
}
\end{center}
\end{table}

\begin{figure}[h]
  \centering
  \includegraphics[width=0.8\hsize]{doc/figures/ex1_graph_hotto_gyagu.png}
  \caption{実験 1 における学習曲線の例}
  \label{fig:ex1_graph}
\end{figure}

\newpage
\changeindent{0cm}
\subsection{実験 2 : マルチモーダルな感情推定の検討}
\changeindent{2cm}

実験 2 では提案手法に則って, マルチモーダルな感情推定の検討を行った.
実験 1 と同様に BERT から得た 768 次元のセリフベクトルと, 入力したセリフが含まれているコマ全体の画像を illustration2vec に入力して得た 4096 次元のコマベクトルを単純に結合させた 4864 次元のベクトルを識別器である 3 層 MLP に入力することでマルチモーダルな感情推定をした.

表 \ref{table:mlp_para} に 3 層 MLP で用いたパラメータ, そして表 \ref{table:ex_para} に学習で用いたパラメータを示す. その他の実験設定は実験 1 と同様にした.


\changeindent{0cm}
\subsection{実験 2 : 結果}
\changeindent{2cm}

表 \ref{table:result_2} に評価用データに対する実験 2 の結果を示す. また, 図 \ref{fig:ex1_graph} に Text Embedding 層に hottoSNS-BERT を用いた場合のギャグタッチにおける学習曲線を示す. (左から Acc, 正例の F 値, loss を表し, 青線が訓練用データ, オレンジの線が検証用データを表す.)

表 \ref{table:result_2} より, Acc に関してはどちらの BERT モデルを用いても
同様にベースラインを超えることができ, 正例の識別率も上がった. また, すべての評価指標において京大 BERT よりも hottoSNS-BERT の方が上回り, 実験 1 および 実験 2 を通してそれぞれ最も高い値を示したことから, マルチモーダルな感情推定の有効性を定量的に確認した. 図 \ref{fig:ex2_graph} において, 訓練用データと検証用データの学習曲線の推移が実験 1 の学習曲線より急勾配で早期に収束しているのは単純に識別器への入力次元数が増えたことに起因すると考えられる. そして, 多くのタッチにおいて実験 1 ではほとんど識別できていなかった正例データが識別できていたことを確認した.

図 \ref{fig:multimodal_add_seikai} にこれらの正例データを示す. 図 \ref{fig:multimodal_add_seikai} より, 文字だけでは人間の目で見ても判断が難しいが, 画像情報 (発話者の顔の表情) を加味することで喜楽寄りのセリフであると理解できるデータであることが分かる.

\newpage

\begin{table}[!ht]
\vspace{10mm}
\begin{center}
\caption{評価用データに対する実験 2 の結果}
\label{table:result_2}
\scalebox{0.45}{
\begin{tabular}{lccccccccccccccc|ccc}
\hline
\multicolumn{1}{c}{\multirow{2}{*}{}} & \multicolumn{3}{c}{ギャグ}                            & \multicolumn{3}{c}{少女漫画}             & \multicolumn{3}{c}{少年漫画}             & \multicolumn{3}{c}{青年漫画}             & \multicolumn{3}{c|}{萌え}               & \multicolumn{3}{c}{5 タッチ総合}         \\
\multicolumn{1}{c}{}                  & Acc                       & P-Recall & P-F 値       & Acc         & P-Recall & P-F 値       & Acc         & P-Recall & P-F 値       & Acc         & P-Recall & P-F 値       & Acc         & P-Recall & P-F 値       & Acc         & P-Recall & P-F 値       \\ \hline
京大 BERT                               & 0.773                     & 0.200    & 0.211       & 0.687       & 0.763    & 0.734       & 0.703       & 0.417    & 0.345       & 0.769       & 0.643    & {\ul 0.545} & 0.641       & 0.500    & 0.489       & 0.715       & 0.583    & 0.546       \\
hottoSNS-BERT                         & {\ul 0.818}               & 0.400    & {\ul 0.400} & {\ul 0.761} & 0.816    & {\ul 0.795} & {\ul 0.781} & 0.500    & {\ul 0.462} & {\ul 0.815} & 0.500    & 0.538       & {\ul 0.703} & 0.545    & {\ul 0.558} & {\ul 0.776} & 0.625    & {\ul 0.622} \\ \hline
ベースライン                                & \multicolumn{1}{l}{0.848} & -        & -           & 0.432       & -        & -           & 0.812       & -        & -           & 0.784       & -        & -           & 0.656       & -        & -           & 0.705       & -        & -
\end{tabular}
}
\end{center}
\vspace{10mm}
\end{table}

\begin{figure}[!h]
  \centering
  \includegraphics[width=0.8\hsize]{doc/figures/ex2_graph_hotto_gyagu.png}
  \caption{実験 2 における学習曲線の例}
  \label{fig:ex2_graph}
\end{figure}

\begin{figure}[!h]
  \centering
  \includegraphics[width=0.75\hsize]{doc/figures/multimodal_add_seikai.png}
  \caption{マルチモーダル化の精度への影響}
  \label{fig:multimodal_add_seikai}
\end{figure}

\newpage
\changeindent{0cm}
\subsection{考察}
\changeindent{2cm}

\changeindent{0cm}
\subsubsection{BERT モデルの合理性}
\changeindent{2cm}

実験結果から, 定量的に京大 BERT よりも hottoSNS-BERT の方が漫画のセリフの分散表現の獲得手法として優れていることが示せた. ここで, 別の視点から京大 BERT と hottoSNS-BERT のどちらがより合理的な分散表現を得られているかについて考察する.

図 \ref{fig:cos_bert} に Data Augmentation によって拡張されたセリフとそれぞれに対応するオリジナルのセリフとのコサイン類似度を京大 BERT と hottoSNS-BERT を用いて計算したヒストグラムを示す. 横軸はコサイン類似度, 縦軸は区間内のデータ数を表している. また, 青色が hottoSNS-BERT, オレンジ色が京大 BERT を表す. ここで, コサイン類似度とは 2 本のベクトルがどれくらい同じ向きを向いているのかを表す指標であり, $-1$ から $1$ までの値を取る. まったく同じ文章同士であればコサイン類似度は $1$ であり, 似た文章であるほど高い値を示す.

\begin{figure}[!h]
  \centering
  \includegraphics[width=0.8\hsize]{doc/figures/cos_bert.png}
  \caption{拡張されたセリフと対応するオリジナルのセリフとのコサイン類似度}
  \label{fig:cos_bert}
\end{figure}

\newpage
図 \ref{fig:cos_bert} より, 一見して京大 BERT の方がコサイン類似度が高い区間に集まっていることから, 優れているようにも思えるが, これは Data Augmentation に大きな問題がないという前提でのみ成立する. 本研究で行った Data Augmentation 手法では, 文法的意味に齟齬が発生しているという問題がある. 例としては, 意味は通じるが形容詞であったものが名詞に置き換わっていたり, 文法的に間違ったものも含まれていた.

表 \ref{table:sogo} に Data Augmentation によって文法的齟齬が生じているデータとこれらのコサイン類似度を京大 BERT と hottoSNS-BERT から出力されたセリフベクトルを用いて計算した結果を示す.

\begin{table}[!h]
\vspace{10mm}
\caption{Data Augmentation による文法的齟齬}
\label{table:sogo}
\centering
\scalebox{1.0}{
\begin{tabular}{ll|cc}
\multicolumn{1}{c}{オリジナルのセリフ} & \multicolumn{1}{c|}{拡張後のセリフ} & 京大 BERT & hottoSNS-BERT \\ \hline
去年は私が着たやつ & 去年は私が着た若者 & 0.97 & -0.19 \\ \hline
そうですか & 正しくですか & 0.89 & -0.19 \\ \hline
\begin{tabular}[c]{@{}l@{}}僕はいいですけど、\\ 気をつけてくださいね\end{tabular} & \begin{tabular}[c]{@{}l@{}}僕はグーですけど、\\ 気をつけてくださいね\end{tabular} & 0.96 & -0.16 \\ \hline
 & \begin{tabular}[c]{@{}l@{}}僕はいいですけど、\\ 真性をつけてくださいね\end{tabular} & 0.94 & 0.25
\end{tabular}
}
\vspace{10mm}
\end{table}

\newpage
\changeindent{0cm}
\subsubsection{i2v でとれている情報は?}
\changeindent{2cm}

\changeindent{0cm}
\subsubsection{結合ベクトルの最適化について}
\changeindent{2cm}

\changeindent{0cm}
\subsubsection{補足 : 人手によるデータセット拡張}
\changeindent{2cm}
