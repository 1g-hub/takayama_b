\newpage
\changeindent{0cm}
\section{数値実験}
\changeindent{2cm}

本章では,数値実験について説明する.

\changeindent{0cm}
\subsection{概要}
\changeindent{2cm}

本研究では, 人工知能を用いた漫画の内容理解のために,
まず自然言語処理を用いて漫画のキャラクタのセリフの感情を推定する.
その上で漫画のコマの画像情報を加えたマルチモーダルな推定手法について検討する.
そして, 実験結果からセリフの感情推定とマルチモーダル化の精度への影響について考察した.

\changeindent{0cm}
\subsection{使用データ}
\changeindent{2cm}

4 コマ漫画ストーリーデータセットには 7 種類の感情ラベル(ニュートラル, 驚愕, 喜楽, 恐怖, 悲哀, 憤怒, 嫌悪)と, アノテーション不備によるラベル不明 (以下, ``UNK" とする) の全 8 種類
が含まれているが, 今回はこの ``UNK" のデータは除いた全 7 種類の感情ラベルが付いたデータのみを扱った. 表 \ref{table:data_ori} に各タッチのオリジナルデータに対する感情ラベルごとのデータ数を示す.

\changeindent{0cm}
\subsection{実験設定}
\changeindent{2cm}

データ数と解析の難しさの問題から, 本研究では 7 種類ある感情ラベルのうち,
喜楽のみを正例クラス, その他の感情ラベルをすべて負例クラスとした
2 クラスのセリフの感情推定を行った.

訓練用データは各タッチの前半 1 話から 5 話までの
Data Augmentation によって拡張されたセリフを用いて, 学習時には $20\%$ を検証用データとした.
また, 評価用モデルは検証用データにおける正例の F 値が最大となるエポックでのパラメータを採用し,
評価用データは後半 6 話から 10 話におけるオリジナルのセリフのみを用いた.
表 \ref{table:data_exp} に各実験で用いた正例と負例のデータ数を示す.



\begin{table}[t]
\begin{center}
\caption{オリジナルデータ数} %タイトルをつける
\label{table:data_ori} %ラベルをつけ図の参照を可能にする
\begin{tabular}{lccccc|l}
\hline
\multirow{2}{*}{感情ラベル} & \multirow{2}{*}{ギャグ} & \multirow{2}{*}{少女} & \multirow{2}{*}{少年} & \multirow{2}{*}{青年} & \multirow{2}{*}{萌え} & \multirow{2}{*}{合計} \\
                       &                      &                     &                     &                     &                     &                     \\ \hline
喜楽                     & 25                   & 77                  & 27                  & 33                  & 47                  & 209 (31.7\%)        \\ \hline
ニュートラル                 & 43                   & 8                   & 55                  & 33                  & 30                  & 169 (25.6\%)        \\
悲哀                     & 25                   & 12                  & 13                  & 16                  & 13                  & 79 (11.9\%)         \\
恐怖                     & 6                    & 11                  & 8                   & 8                   & 9                   & 42 (6.3\%)          \\
憤怒                     & 4                    & 5                   & 2                   & 7                   & 2                   & 20 (3.0\%)          \\
嫌悪                     & 2                    & 4                   & 3                   & 3                   & 4                   & 16 (2.4\%)          \\ \hline
UNK                    & 7                    & 3                   & 5                   & 2                   & 6                   & 23 (3.4\%)          \\ \hline
合計                     & 131                  & 136                 & 130                 & 131                 & 131                 & 659
\end{tabular}
\end{center}
\end{table}

\begin{table}[b]
\begin{center}
\caption{実験で用いたデータ数} %タイトルをつける
\label{table:data_exp} %ラベルをつけ図の参照を可能にする
\begin{tabular}{llccccc|l}
\hline
\multirow{2}{*}{}       & \multirow{2}{*}{感情ラベル} & \multirow{2}{*}{ギャグ} & \multirow{2}{*}{少女} & \multirow{2}{*}{少年} & \multirow{2}{*}{青年} & \multirow{2}{*}{萌え} & \multirow{2}{*}{合計} \\
                        &                        &                      &                     &                     &                     &                     &                     \\ \hline
\multirow{2}{*}{訓練用データ} & 喜楽                     & 1115                 & 2672                & 940                 & 999                 & 1766                & 7492 (37.1\%)       \\
                        & その他                    & 2766                 & 1396                & 3077                & 3146                & 2324                & 12709 (62.9\%)      \\ \hline
\multirow{2}{*}{評価用データ} & 喜楽                     & 10                   & 38                  & 12                  & 14                  & 22                  & 96 (29.5\%)         \\
                        & その他                    & 56                   & 29                  & 52                  & 51                  & 42                  & 230 (70.5\%)
\end{tabular}
\end{center}
\end{table}

\newpage
\changeindent{0cm}
\subsection{実験 1}
\changeindent{2cm}

実験 1 では, セリフ 1 文のみを入力とする感情推定を行った.
