\newpage
\changeindent{0cm}
\section{はじめに}
\changeindent{2cm}
近年, 深層学習を始めとする機械学習技術の大きな発展を受けて, 人工知能を用いた創作物理解が注目されている.
しかし, 創作は高次の知的活動であるため, いまだに実現が困難なタスクである.
人の創作物の理解に関する分野の中でもコミック工学 \cite{comic} など漫画を対象とした研究は,
絵と文章から構成される漫画を対象とするため, 自然言語処理と画像処理の両方の側面を持つ
マルチモーダルデータを扱う分野である.
コミック工学の分野では様々な研究が報告されているが,
その多くは画像処理に基づいた研究であり,
自然言語処理による内容理解を目指した研究は少ない.
その大きな原因のひとつとしてデータが十分ではないという点が挙げられる.
また, 漫画に含まれるテキストには, 口語表現, 擬音語, 表記揺れといった漫画特有の言語表現を含み,
これらの扱いについて考慮する必要がある.
そして, 漫画が著作物であることに起因する研究用データの不足も課題となっている.

本研究では人工知能を用いた漫画の内容理解のために,
漫画におけるキャラクタのセリフのマルチモーダルな感情推定を目的とする.
まず自然言語処理を用いた漫画のセリフの感情を推定して,
その上で漫画のコマの画像情報を加えたマルチモーダル化について検討する.

以下に本論文の構成を示す.
まず, 2 章ではコミック工学に関連するデータセットについて,
また 3 章では本研究で用いる要素技術について概説する.
% 次に, 4 章で関連研究について概観する.
次に, 4 章では
漫画のセリフのマルチモーダルな感情推定を行うための提案手法について述べる.
そして, 5 章において, 実験手法とその考察を示す.
最後に, 6 章で本研究の成果をまとめた上で, 今後の課題について述べる.
