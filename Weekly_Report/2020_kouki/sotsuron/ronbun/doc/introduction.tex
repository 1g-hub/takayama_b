\newpage
\changeindent{0cm}
\section{はじめに}
\changeindent{2cm}
近年, 深層学習を始めとする機械学習技術の大きな発展を受けて, 人工知能を用いた創作物理解が注目されている.
しかし, 創作物理解や作品の自動生成といった試みは工学的に興味深く意義が大きい反面,そもそも人の創作物理解は高次の知的活動であり,どういったタスクであれば人工知能が創作物を理解したと言えるのかを定義することさえ現状では難しい.
人の創作物の理解に関する分野の中でもコミック工学 \cite{comic} など漫画を対象とした研究は,
絵と文章から構成される漫画を対象とするため, 自然言語処理と画像処理の両方の側面を持つ
マルチモーダルデータを扱う分野である.
コミック工学の分野では様々な研究が報告されているが,
その多くは画像処理に基づいた研究であり,
自然言語処理による内容理解を目指した研究は少ない.
その大きな原因のひとつとして, 漫画が著作物であることに起因するデータ不足が挙げられる.
また, 漫画に含まれるテキストには, 口語表現, 擬音語, 表記揺れといった漫画特有の言語表現を含み,
これらの扱いについて考慮する必要がある.
漫画という媒体の情報を十分に活用するにはマルチモーダルな解析が最も良いと考えられ, その中でも十分な意義があると考えられるタスクはセリフの感情推定である. その理由としては 2 つある. 人工知能を用いた対話型システムの精度向上のためには自然な表情や発言, 振る舞いから人の潜在的な感情を推定できるようになる必要があること. そして, 創作者に対して作品の展開を汲んだ適切なセリフの自動生成といった制作時間の削減や質の向上に関わる創作支援に繋がるからである.

したがって, 本研究では人工知能を用いた漫画の内容理解のために,
漫画におけるキャラクタのセリフのマルチモーダルな感情推定を目的とする.
まず自然言語処理を用いた漫画のセリフの感情を推定して,
その上で漫画のコマの画像情報を加えたマルチモーダル化について検討する.

以下に本論文の構成を示す.
2 章ではコミック工学に関連するデータセットについて説明し,
3 章では本研究で用いる要素技術について概説する.
4 章では 4 コマ漫画に関連する従来研究について概観する.
5 章では漫画のセリフのマルチモーダルな
感情推定のための提案手法とシソーラスを用いたデータ拡張について述べる.
そして, 6 章において実験とその考察を示す.
最後に 7 章で本研究の成果をまとめた上で, 今後の課題について述べる.
