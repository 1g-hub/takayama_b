\newpage
\changeindent{0cm}
\section{4 コマ漫画に関する従来研究}
\changeindent{2cm}

本章では, コミックに関する研究について, 4 コマ漫画の研究事例とデータセットについて述べる.

4 コマ漫画は代表的な漫画形式のひとつであり,4 つのコマ (齣) によって完結した短い話を表現する.
4 コマ漫画の基本的なストーリー展開は,各コマを最初から順に起承転結に対応
させ,結に相当する最終コマをオチとする場合が多い. それ以外にも 3 コマを序破急に対
応させる場合や,2 段オチ, オチを必ずしも必要としないストーリー 4 コマなどが存在する.
4 コマ漫画に関する研究としては,
4 コマにおける画像特徴が与える感情識別に関する研究 \cite{ueno-emotion2016} や
ストーリー理解過程の解析研究 \cite{ueno-oti2017},
4 コマ漫画ではないが既存漫画のデータを利用した
2 コマ漫画の生成に関する研究 \cite{jsai18mukaeyama} が報告されている.
また 4 コマ漫画の自動生成に関する研究 \cite{ueno:dcai2016}
や遺伝的アルゴリズムに基づく感性解析に
4 コマ漫画を用いた研究 \cite{GA4koma} もなされている.

また,ストーリーに関しては 4 コマ漫画の内容に踏み込んだ研究として,
コマの順序識別に関する研究 \cite{ueno2016estimation,jsai18fujino} が報告されている.
しかしながら手法,データセットともにまだ十分とは言えず,今後の発展が期待されている分野である.
